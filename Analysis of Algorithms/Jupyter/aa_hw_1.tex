%%%%%%%%%%%%%%%%%%%%%%%%%%%%%%%%%%%%%%%%%%%%%%%%%%%%%%%%%%%%%%%%%%
%
% Michael Scherger
%
% Analysis of Algorithms
%
% Homework Assignment #1
%
%%%%%%%%%%%%%%%%%%%%%%%%%%%%%%%%%%%%%%%%%%%%%%%%%%%%%%%%%%%%%%%%%%
%%%%%%%%%%%%%%%%%%%%%%%%%%%%%%%%%%%%%%%%%%%%%%%%%%%%%%%%%%%%%%%%%%
%
% Score Card and Answer Sheets
%
%%%%%%%%%%%%%%%%%%%%%%%%%%%%%%%%%%%%%%%%%%%%%%%%%%%%%%%%%%%%%%%%%%
\documentclass[11pt]{article}
\usepackage{clrscode4e}
\usepackage{tcucosc}
\usepackage{fancyhdr}
\usepackage{geometry}
\usepackage[addpoints]{exam}
\usepackage{ulem}


%%%%%%%%%%%%%%%%%%%%%%%%%%%%%%%%%%%%%%%%%%%%%%%%%%%%%%%%%%
%%%
%%% Set Page Size
%%%
%%%%%%%%%%%%%%%%%%%%%%%%%%%%%%%%%%%%%%%%%%%%%%%%%%%%%%%%%%
\geometry{hmargin={1.0in,1.0in},vmargin={1.0in,1.0in}}


%%%%%%%%%%%%%%%%%%%%%%%%%%%%%%%%%%%%%%%%%%%%%%%%%%%%%%%%%%%
%%%
%%% Fancy Header
%%%
%%%%%%%%%%%%%%%%%%%%%%%%%%%%%%%%%%%%%%%%%%%%%%%%%%%%%%%%%%
\pagestyle{fancy}
\fancyhf{}
\fancyhead{}
\fancyfoot{}
\lhead{COSC 40403}
\chead{Analysis of Algorithms}
\rhead{Homework 1}
\lfoot{Fall 2018}
\cfoot{Version: 08272018-001}
\rfoot{\thepage}


%%%%%%%%%%%%%%%%%%%%%%%%%%%%%%%%%%%%%%%%%%%%%%%%%%%%%%%%%
%%%
%%% Renew Commands
%%%
%%%%%%%%%%%%%%%%%%%%%%%%%%%%%%%%%%%%%%%%%%%%%%%%%%%%%%%%%%
\renewcommand{\headrulewidth}{0.4pt}	% line width for fancy header
\renewcommand{\footrulewidth}{0.4pt}	% line width for fancy footer
\renewcommand\refname{}					% for bibliography


%%%%%%%%%%%%%%%%%%%%%%%%%%%%%%%%%%%%%%%%%%%%%%%%%%%%%%%%%%%%%%%%%%
%
% Begin Document
%
%%%%%%%%%%%%%%%%%%%%%%%%%%%%%%%%%%%%%%%%%%%%%%%%%%%%%%%%%%%%%%%%%%
\begin{document}
\pagestyle{empty}

\noindent{\large\bfseries Name: PUT YOUR NAME HERE{\hrulefill}}\\
\noindent{\large\bfseries COSC 40403 - Analysis of Algorithms: Fall 2019: Homework 1}\\
\noindent{\large\bfseries Due: 23:59:59 on September 4}


%%%%%%%%%%%%%%%%%%%%%%%%%%%%%%%%%%%%%%%%%%%%%%%%%%%%%%%%%%%%%%%%%%
%
% Score Card and Answer Sheets
%
% Comment out one-or-the-other to show or not-show the answers.
%
%%%%%%%%%%%%%%%%%%%%%%%%%%%%%%%%%%%%%%%%%%%%%%%%%%%%%%%%%%%%%%%%%%
\printanswers
%%\noprintanswers


%%%%%%%%%%%%%%%%%%%%%%%%%%%%%%%%%%%%%%%%%%%%%%%%%%%%%%%%%%%%%%%%%%
%
% Score Card
%
%%%%%%%%%%%%%%%%%%%%%%%%%%%%%%%%%%%%%%%%%%%%%%%%%%%%%%%%%%%%%%%%%%
\ifprintanswers
\noindent
\begin{center}
	\gradetable[v][questions]
\end{center}
%\newpage
\bigskip
\fi


%%%%%%%%%%%%%%%%%%%%%%%%%%%%%%%%%%%%%%%%%%%%%%%%%%%%%%%%%%%%%%%%%%
%
% Question 
%
%%%%%%%%%%%%%%%%%%%%%%%%%%%%%%%%%%%%%%%%%%%%%%%%%%%%%%%%%%%%%%%%%%
\begin{questions}
\question[10]
Consider the {\bf {\em All-Peaks problem}}:\\
{\bf Input: } A sequence of $n$ integers $A=\langle a_1,  \dots, a_n\rangle$.  The size of $A$ may be 0, 1, or $n$.\\
{\bf Output:} A sequence that contains indices of all the peaks in the sequence. The size of the returned sequence may be 0.  \\\\
Write pseudocode for \proc{All-Peaks}, which scans through the sequence, looking for peaks.  A peak is defined to be a data element that is strictly greater than its adjacent data elements.  If a peak is found, append the peak index to a result list which is returned from the function when your algorithm completes.\\
Here are some examples:
\begin{verbatim}
[5, 10, 20, 15] -> [2]
[10, 20, 15, 2, 23, 90, 67] -> [1, 5]
[] -> []
[1] -> [0]
[1, 2] -> [1]
[2, 1] -> [0]
[2, 1, 2] -> [0, 2]
[1, 1] -> []
[2, 2, 1, 1, 3, 3, 1, 1, -1, -1, 0] -> [10]
\end{verbatim}
Develop a cost-model equation, $T(n)$, for the running time for \proc{All-Peaks} similar to what was developed in class and in our textbook.  Express your solution in terms of $\Theta$-notation.  Justify your answers.
\begin{solutionorbox}\\ \\
	\begin{tabular}{l c c}
	    All\textunderscore Peaks(A) & {\it cost} & {\it times}\\
	    1\quad If A.length = 0, return [\space] & c\textsubscript{1} &  1\\
	    2\quad If A.length = 1, return [0] & c\textsubscript{2} &  1\\
	    3\quad If i == 0 and A[i] $>$ A[i+1] append i to the result list & c\textsubscript{3} &  1\\
	    4\quad Else if i+1 == A.length - 1 and i $>$ i - 1 append i to the result list & c\textsubscript{4} &  1\\
	    5\quad If neither case 1 or 2 has been met, loop {\bf for} i = 0 to A.length & c\textsubscript{5} &  n\\
	    6\qquad if i $>$ i - 1 and i $>$ i + 1 append i to the result list & c\textsubscript{6} &  n - 1\\
	    7\quad Return the result list & c\textsubscript{7} & 1\\
	\end{tabular}\\
	The best case scenario for the algorithm is when A.length is less than or equal to 1 because the function returns nearly immediately, taking constant time.\\
	If the algorithm reaches the for loop, it takes linear (n) time.\\
	T(n) = c\textsubscript{1} + c\textsubscript{2} + c\textsubscript{3} + c\textsubscript{4} + c\textsubscript{5}(n) + c\textsubscript{6}(n-1) + c\textsubscript{7}\\
	\null\qquad\space = (c\textsubscript{5} + c\textsubscript{6})n - (c\textsubscript{6}) + (c\textsubscript{1} + c\textsubscript{2} + c\textsubscript{3} + c\textsubscript{4} + c\textsubscript{7})\\
	\null\qquad\space = an + b for constants a and b\\
	T(n) = $\Theta$(n)
\end{solutionorbox}

\ifprintanswers
\newpage
\else
\bigskip
\fi


%%%%%%%%%%%%%%%%%%%%%%%%%%%%%%%%%%%%%%%%%%%%%%%%%%%%%%%%%%%%%%%%%%
%
% Question
%
%%%%%%%%%%%%%%%%%%%%%%%%%%%%%%%%%%%%%%%%%%%%%%%%%%%%%%%%%%%%%%%%%%
\question[15]
Using your algorithm developed in the preceding question, write a Jupyter notebook using the Python programming language that implements your algorithm.  Develop a main function that times the execution of your \proc{All-Peaks} function.
\begin{enumerate}
	\item [6 pts] Write a program in Jupyter notebook that implements your algorithm.
	\item [2 pts] Test the correctness of your algorithm with the examples shown in the previous problem.
	\item [2 pts] Time your \proc{All-Peaks} function for sequence sizes $[1..250]$.  
	\item [2 pts] Using a spreadsheet program (e.g. Excel), import your timing data and make a plot of time vs. n.  Include the chart in your solution.  Or use some of the Python data plotting methods and include a plot in your Python notebook.
	\item [3 pts] Using the functions built into your spreadsheet program, find the slope and y-intercept for your data.  What is your run-time equation $T(n)$?  (Write out your equation.)
\end{enumerate}
\begin{solutionorbox}
	Place solution in Python notebook \ttfamily{h1q2.ipynb}.
\end{solutionorbox}

\ifprintanswers
%\newpage
\bigskip
\else
\bigskip
\fi


%%%%%%%%%%%%%%%%%%%%%%%%%%%%%%%%%%%%%%%%%%%%%%%%%%%%%%%%%%%%%%%%%%
%
% Question
%
%%%%%%%%%%%%%%%%%%%%%%%%%%%%%%%%%%%%%%%%%%%%%%%%%%%%%%%%%%%%%%%%%%
\question[30]
Look up the algorithm for \proc{Selection-Sort} on the Internet or textbook.  
\begin{enumerate}
	\item [7 pts] Develop a cost-model equation for the running time for \proc{Selection-Sort} similar to what was developed in class and in our textbook.  SHOW YOUR WORK!!!!!!
	\item [3 pts] How many comparisons are made in the worst-case using \proc{Selection-Sort}?  Express your answers in $\Theta$-notation.
	\item [3 pts] How many data exchanges are made in the worst-case using \proc{Selection-Sort}?  Express your answers in $\Theta$-notation.
	\item [7 pts] Implement your algorithm in a Python notebook.
	\item [7 pts] Implement the \proc{Merge-Sort} algorithm in your Python notebook.
	\item [3 pts] Using a random list of 1000 integers, write a main function that determines at what value of $n$ \proc{Merge-Sort} is faster than \proc{Selection-Sort}.
\end{enumerate}
\begin{solutionorbox}
	Place solution in Python notebook \ttfamily{h1q3.ipynb}.
\end{solutionorbox}


%%%%%%%%%%%%%%%%%%%%%%%%%%%%%%%%%%%%%%%%%%%%%%%%%%%%%%%%%%%%%%%%%%
%
%%%%%%%%%%%%%%%%%%%%%%%%%%%%%%%%%%%%%%%%%%%%%%%%%%%%%%%%%%%%%%%%%%
\end{questions}
\end{document}
